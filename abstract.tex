\section{summary}
Elastic reflection waveform inversion (ERWI) utilize the reflections to update the low and
intermediate wavenumbers in the
%deeper part of model, which can provide good initial models for EFWI. However, ERWI suffers from
deeper part of model. However, ERWI suffers from
the cycle-skipping problem due to the objective function of waveform residual. Since traveltime
information relates to the background model more linearly, 
%the nonlinearity of traveltime
%fitting will be lower 
%compared with amplitude fitting. Thus, we use the $L_2$ norm of the traveltime
we use the traveltime
residuals as objective function to 
update background velocity model using
wave equation reflected traveltime inversion (WERTI). 
The reflection kernel analysis shows that mode decomposition can suppress the artifacts in
gradient calculation. 
We design a two-step inversion strategy, in which PP reflections are firstly used to invert P wave
velocity ($V_p$),
followed by S wave velocity ($V_s$) inversion with PS reflections. %based on the well recovered $V_p$.
P/S separation of multi-component seismograms and spatial wave mode decomposition %provide P or S wave data effectively, which helps
can reduce the nonlinearity of inversion effectively by selecting suitable P or S wave subsets for
hierarchical inversion.
Numerical example of Sigsbee2A model validates the effectiveness of the 
algorithms and strategies for elastic WERTI (E-WERTI).

