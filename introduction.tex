\section{Introduction}
With the development of high-performance computational abilities, people are paying more
attention to the elastic full waveform inversion (EFWI) to recover the elastic properties of the
subsurface \cite[]{vigh:2014}.
EFWI provides high-resolution model estimation but notoriously surfers from
the same nonlinearities or cycle-skipping problems as in acoustic
case and also the trade-offs of multi-parameter inversion \cite[]{operto2013guided}.
In the absence of low frequency data and/or good initial models, EFWI falls into local minimal
easily
because of its incapability to recover the low and intermediate wavenumber components of the model.

\cite{xu:2012} suggested using a reflection waveform inversion (RWI) method to suppress the
nonlinearity in FWI, 
which aim to invert the long-wavelength components of the model by using the reflections 
predicted by migration/demigration process.
%The key point of RWI is to project the data-domain residuals back to the reflection wavepath to 
RWI highly relies on the accurate reflectivity to generate the reflections that can match the
observed data. However, it is very challenging and also expensive to obtain a good reflectivity model through 
least-square migration when initial model is far away from the true value.
Through minimizing misfit function of waveform in data domain, the RWI method is developed
by several works \cite[]{Wu2015b,Zhou2015}, and recently extended to elastic case by
\cite{Guo2016}.
%is to project the data-domain residuals back to the reflection wavepath to 
The misfit function also can be built in image domain in the manner of wave equation migration
velocity analysis (WEMVA), which tries to maximize
energy at zero offset location in the extended image space
\cite[]{BiondiEtAl2012,SunEtAl2012}.
\cite{RaknesEtAl2016} 
developed the image-domain method in the 3D elastic media. 
\cite{Wang2017WEMVA} utilized PS image to update the $V_s$ model through mode decomposition.
But it is a big issue that the extended domain approach requires huge computational cost,
especially in 3D case.

%for both the imaging condition and the forward modelling in each iteration.

%\cite{Zhou2015} proposed a joint FWI method to combine the diving and reflected waves to
%utilize both the conventional FWI and RWI.
%In addition, \cite{Wang:2015} found that wave mode decomposition may be beneficial to deal with the
%elastic parameter trade-offs.
Traveltime information are more sensitive and linearly related to
the low-wavenumber components of the model. Therefore, traveltime inversion will be more robust and helpful to
build initial models containing long-wavelength components for
%build appropriate initial models containing long-wavelength components for
conventional FWI \cite[]{Chi2015, Luo2016}.
\cite{Ma2013} introduced a wave equation reflected traveltime inversion (WERTI) method to build the low
wavenumber of the model. 
Unfortunately, in elastic case, traveltimes of a particular wave mode are difficult
to extract due to the complicated mode-conversions. The multi-parameter trade-offs will increase the
nonlinearity as well. \cite{WangEtAl2017} obtained good results by utilizing the wave mode decomposition to precondition the
gradients in EFWI. The mode decomposition has been proved an efficient tool to provide decomposed data for
hierarchical strategies.

%Elastic reflections carry background information of the P and S wave velocities can
%be in favor of EFWI with good initial P and S wave
%velocity models, 
%In this abstract,   
%we aim to tackle the traveltime misfits of P-P and P-S reflections with the help
%of wave mode decomposition and dynamic image warping (DIW) \cite[]{Hale2013}.
%In this abstract, we separate the PP and PS reflections in the observed and predicted multicomponent data to
%extract the individual traveltime residuals through DIW \cite[]{Hale2013}.
%using surface P/S wave
%separation. Dynamic image warping \cite[]{Hale2013} helps to extract the local traveltime residual
%between the observed and the predicted data.
We calculate the reflection wavepath and its components of different wave modes in elastic media.
The investigation of reflection wavepath (kernel) shows the effectiveness of mode decomposition to suppress the artifacts
in the gradient calculation.
P/S separation of multicomponent seismograms is applied to the observed and predicted reflection data 
to extract the individual traveltime residuals through DIW \cite[]{Hale2013}.
Based on the analysis of elastic reflection kernels and the separated traveltime residuals, 
we design a two-stage workflow to implement
the E-WERTI method,
in which the traveltime of PP is firstly used to recover the background
$V_p$ model followed by inverting the background $V_s$ model through the traveltime of PS
reflections.
During the $V_s$ inversion, we precondition the $V_s$ gradient through spatial wave mode
decomposition.
Finally, the numerical example of Sigsee2A model proves the robustness and validity of our
E-WERTI method.
